%
% Copyright (c) 2020 Antonio Coín Castro
%
% This work is licensed under a
% Creative Commons Attribution-ShareAlike 4.0 International License.
%
% You should have received a copy of the license along with this
% work. If not, see <http://creativecommons.org/licenses/by-sa/4.0/>.

Este trabajo se compone de dos partes claramente diferenciadas, cada una enfocada principalmente a una disciplina distinta y con un tema de estudio diferente. A continuación resumimos con cierto detalle el contenido de cada una de estas partes.

En la Parte \ref{part:cs} se recoge el contenido más aplicado y relacionado con la ingeniería informática, si bien no por ello carece de contenido matemático. El objetivo principal de esta parte del trabajo es estudiar la interacción entre dos campos importantes en la informática y sus aplicaciones, como son los sistemas difusos y la plataforma Big Data.

Por un lado, los sistemas difusos son modelos de aprendizaje basados en las herramientas de la lógica difusa y la teoría de conjuntos difusos, que introduce un enfoque novedoso en la manera de modelar muchos problemas. Específicamente, en este contexto las entidades fundamentales son los \textit{conjuntos difusos}, una generalización de los conjuntos usuales mediante la cual se permite que los elementos tengan un cierto grado de pertenencia al conjunto, en lugar de simplemente pertenecer o no pertenecer a él.
Utilizando estos conjuntos como base se construye una lógica multivaluada que permite permite tratar y razonar con situaciones que presentan una ambigüedad inherente y son difíciles de analizar mediante la lógica bivalente a la que estamos acostumbrados, como puede ser el procesamiento del lenguaje humano.

Por otro lado, de un tiempo a esta parte hemos podido observar cómo la cantidad de datos que manejamos ha ido aumentando a un ritmo frenético. La mayoría de ordenadores personales que se comercializan en la actualidad ofrecen una capacidad de almacenamiento en el rango de los \textit{terabytes}, y para aplicaciones específicas de tratamiento y análisis de datos estas cifras crecen considerablemente hasta los \textit{petabytes} ($10^3$ \textit{terabytes}) o los \textit{exabytes} ($10^6$ \textit{terabytes}). Es por eso que se ha vuelto necesario disponer de una infraestructura y unos algoritmos que permitan soportar y tratar adecuadamente grandes cantidades de datos.

Es aquí donde entran en juego los sistemas difusos o sistemas basados en reglas difusas del tipo \textit{si-entonces}, que pueden ser entendidos como sistemas basados en el conocimiento preparados para realizar inferencia, extraer conclusiones sobre unos datos y resolver multitud de problemas que presenten ambigüedad o imprecisión en su planteamiento o resolución, entre los que destacan los problemas de control automático. El hecho de disponer de una gran cantidad de datos para confeccionar modelos de aprendizaje de datos en estos sistemas hace que la calidad de la respuesta sea superior, permitiendo en muchos casos resolver problemas de manera aproximada pero suficientemente buena en un tiempo reducido. Además, la interpretabilidad que proporcionan las reglas difusas hace que el resultado final tenga más valor, pues se puede explorar el procedimiento por el cual se llega a unas ciertas conclusiones.

En este trabajo realizamos primeramente un estudio desde el punto de vista teórico de los conjuntos difusos, la lógica difusa y los sistemas difusos. También estudiamos las características fundamentales en torno al concepto de Big Data y la infraestructura, filosofía de los algoritmos y \textit{frameworks} que han ido surgiendo en respuesta al incremento en la importancia de este campo. En particular, discutimos el paradigma de programación MapReduce y su implementación en Apache Spark. Una vez hecho esto, planteamos un problema de diseño e implementación de algoritmos difusos escalables, que pongan de manifiesto la afinidad comentada anteriormente entre los sistemas difusos y el campo de Big Data. Cabe destacar que no perseguimos realizar una mera paralelización de algoritmos existentes, sino que pretendemos estudiar y proponer diseños alternativos que permitan una mayor escalabilidad.

Finalmente, realizamos un estudio comparativo de los algoritmos desarrollados, ejecutándolos en un entorno distribuido con suficiente capacidad como para manejar de forma efectiva conjuntos formados por varios millones de puntos. Comparamos estos algoritmos con algunos otros modelos escalables no difusos ya implementados en Spark, y analizamos los resultados y las conclusiones que se pueden extraer sobre los mismos.



\EndParagraph
En la Parte \ref{part:maths} encontramos el contenido puramente matemático del trabajo. En ella se acomete el estudio de ecuaciones diferenciales de primer orden en forma implícita, es decir, aquellas para las que la derivada no se puede despejar. En un curso básico de ecuaciones diferenciales ordinarias usualmente se trata siempre el caso explícito, esto es, ecuaciones de la forma $y'=f(x,y)$, y no se suelen analizar con detalle las ecuaciones implícitas de la forma $F(x,y,y')=0$. En este trabajo proponemos profundizar en el estudio de estas últimas ecuaciones desde el punto de vista de la teoría general de aplicaciones diferenciables, analizando los comportamientos que presentan este tipo de ecuaciones y aportando condiciones generales para su tratamiento. En concreto, nos interesan especialmente cuestiones relacionadas con la existencia y unicidad de soluciones, pues en este caso no podemos aplicar los teoremas usuales en este aspecto.

En primer lugar comenzamos con una exposición de la teoría básica de ecuaciones diferenciales ordinarias, en un esfuerzo por hacer que el trabajo sea más autocontenido y fijar la notación que se usará a lo largo del mismo. Se definen con rigor los conceptos de ecuaciones diferenciales y soluciones de las mismas. También nos detenemos en analizar con detalle la interpretación geométrica que tienen las ecuaciones diferenciales en forma explícita, y cómo producen en el plano un campo de direcciones que marca la trayectoria de las soluciones o curvas integrales. Tras ello, planteamos la cuestión de existencia y unicidad de soluciones, proporcionando ejemplos en los que no se verifica ninguna de estas, y enunciando los resultados clásicos que establecen condiciones para su tratamiento, como son el teorema de Peano o el teorema de Picard-Lindelöf. En ambos casos se refiere al lector a fuentes donde puede consultar la demostración, tanto en su forma original como en términos más modernos.

Tras esta primera toma de contacto, pasamos a definir las ecuaciones diferenciales en forma implícita, entendiéndolas como ecuaciones de la forma $F(x,y,p)=0$, donde $F$ es una función de $\R^3$ en $\R$ y $p=dy/dx$. En primer lugar analizamos cuáles son las condiciones que nos permiten transformarlas localmente en ecuaciones explícitas, para poder aplicar toda la teoría desarrollada anteriormente. Estas condiciones resultan ser ni más ni menos que aquellas que permiten aplicar el \textit{teorema de la función implícita} para expresar $p$ como función de $x$ e $y$ (en concreto, que la derivada parcial de $F$ con respecto a su tercera variable no se anule). Tras ello, estudiamos un primer enfoque algebraico en el que se discute cuándo podemos asegurar que por un punto del plano pasan un número finito de soluciones. En este caso no podemos generalmente aspirar a tener una única solución en cada punto, pues en general la ecuación $F=0$ representa la superposición de varios campos de direcciones, cada uno de ellos correspondiente a una familia de soluciones.

Siguiendo con el estudio de las ecuaciones implícitas, cambiamos el enfoque y nos planteamos el problema desde un punto de vista geométrico, entendiendo la ecuación $F=0$ como una superficie $M=F^{-1}(0)$ vista en un espacio de tres dimensiones. En este apartado hacemos uso de varios conceptos y resultados de geometría diferencial, y en particular, de geometría diferencial de superficies. Estudiamos cómo surge en la superficie $M$ un campo de direcciones, que al proyectarse en el plano produce localmente un campo de direcciones para cada una de las ecuaciones representadas en la expresión $F=0$. Además, este planteamiento geométrico nos permite aproximarnos de forma efectiva al estudio de puntos singulares y soluciones singulares que surgen en este tipo de ecuaciones, definiendo los conceptos de \textit{curva criminante} y \textit{curva discriminante}. En este sentido presentamos un resultado interesante que asegura que en torno a puntos singulares que cumplan una cierta condición de regularidad, podemos expresar mediante un adecuado cambio de variable la ecuación $F=0$ en la denominada \textit{ecuación implícita en forma normal}, mucho más sencilla de resolver.

Finalmente concluimos la exposición examinando una serie de ecuaciones clásicas en el ámbito de las ecuaciones en forma implícita, como son las ecuaciones de Clairaut y de Lagrange. En cada caso estudiamos cómo se aplica la teoría desarrollada en el trabajo para el análisis de dicha ecuación, proporcionando también métodos efectivos de resolución y visualizando ejemplos concretos de algunas curvas integrales y sus peculiaridades.\\

\noindent\textsc{palabras clave:} sistemas difusos, computación escalable, Big Data, ecuaciones implícitas, aplicaciones diferenciables, singularidades.
