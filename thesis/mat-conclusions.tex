%
% Copyright (c) 2020 Antonio Coín Castro
%
% This work is licensed under a
% Creative Commons Attribution-ShareAlike 4.0 International License.
%
% You should have received a copy of the license along with this
% work. If not, see <http://creativecommons.org/licenses/by-sa/4.0/>.

Throughout this document we have presented the theory of first order implicit differential equations, exploring the classical texts related to them and establishing general conditions for their treatment. We feel that our initial goal has been accomplished, since a fairly complete study has been developed. In addition, some interesting examples have been presented, which illustrate how this theory can be applied to solve specific equations in a satisfactory way.

The theory of differential equations not solved for the derivative goes back centuries, to the time of Newton and Leibniz, but the written materials discussing them in modern terms are scarce. A significant contribution of our work has been to put together those resources and unify them in a complete theory that tackles the principal aspects of these equations. When possible, we have included the references of the original books and articles where a certain technique or equation first appeared, and in doing so we have obtained a collection of sources in which a treatment of this type of equations is considered, which can be useful for further research.

Another consequence of the unification mentioned earlier is that different points of view are presented when it comes to dealing with implicit equations. In the first part of the work we explored a more algebraic approach to solving these equations, while later we moved on to a very geometrical interpretation of them. In the latter approach a significant amount of concepts from differential geometry and the geometry of surfaces are studied as tools for modelling implicit equations, and they can be complex in the sense that they require a firm understanding of the underlying theory and its implications. We paid special attention to how the definitions and results were stated so that they would be consistent with every part of the theory developed, which again was not an easy task due to the lack of a consolidated work that served as a reference for every part. A relevant example of this situation are singular solutions, as their precise definition varies from author to author.

In regard to future areas of research and extensions of this work, several paths are possible. Firstly, we could look at the generalization of the theory to second order implicit ODEs, which are usually utilized to model and describe the behaviour of many systems in physics and other sciences. A good reference for this would be \cite{takahashi2007implicit}.

Secondly, we could delve deeper into the matter of singular solutions, for example studying the behaviour of the equation near a singular point, distinguishing whether or not there is a whole neighbourhood of singular points, as done in \cite{rabier1989implicit}.

Lastly, we could also study other classical examples of equations not solved for the derivative, analyzing the changes of variables needed to efficiently solve them or even provide a strategy to tackle equations in specific forms, such as $y'=f(y,y')$ or $F(y,y')=0$. It is worth noting that many implicit equations arise in the field of calculus of variations when looking for critical points of functionals of one variable, since normally the necessary conditions imposed by the Euler-Lagrange equations are given in implicit form. This is the case of the problem of finding the curve of fastest descent, the \textit{brachistochrone}, in which the differential equation involved is given by $y(1+(y')^2)=C^2$.

In conclusion, a sufficiently deep study of implicit differential equations has been carried out, focusing mostly in the theoretical aspects but also providing relevant examples that have been largely studied by the mathematical community. Furthermore, this theory can serve as a starting point for further research on this topic, from which many practical applications are sure to emanate as well.
