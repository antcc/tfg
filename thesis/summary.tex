%
% Copyright (c) 2020 Antonio Coín Castro
%
% This work is licensed under a
% Creative Commons Attribution-ShareAlike 4.0 International License.
%
% You should have received a copy of the license along with this
% work. If not, see <http://creativecommons.org/licenses/by-sa/4.0/>.

In the first part of this document we present a study of the interaction between fuzzy systems and Big Data, both from a theoretical and an applied point of view. We begin by studying the fundamentals of fuzzy set theory and fuzzy logic, stopping to analyze in depth the concepts of fuzzy \textit{if-then} rules and fuzzy reasoning. Next, we delve into the question of what Big Data is and we explore its fundamental characteristics, reviewing the state of the art in the infrastructures and algorithms in this field and examining the industry standard frameworks: the MapReduce paradigm and the Apache Spark platform. Once this theoretical study is complete, we undertake an implementation task to design and program a suite of scalable fuzzy learning algorithms, explaining how they could be useful in solving problems that imply large amounts of data. Finally, we perform a comparative study of the algorithms developed, executing them in a distributed environment and analyzing the results obtained.

In the second part of this work we carry out a study of first order implicit differential equations, that is, ordinary differential equations that are not solved for the derivative and cannot be directly put in explicit form. We begin with a review of the basics concepts in the theory of explicit differential equations, recalling their geometrical interpretation and the main theorems for establishing the existence and uniqueness of solutions. Next we introduce implicit differential equations both from an algebraic and a geometric point of view, framed within the general theory of differentiable mappings. We explore general conditions that allow us to treat these equations locally as explicit equations, while also analyzing the singular solutions that may arise in their resolution. Lastly, we present several concrete examples of classical equations that have been studied in the mathematical community, relating them to the theory developed in this work and providing general methods for solving them.\\


\noindent\textsc{keywords:} fuzzy systems, computational scalability, Big Data, implicit equations, differentiable mappings, singularities.
