\let\ifdeutsch\iffalse
\let\ifenglisch\iftrue
\input{include/pre-documentclass}
\documentclass[
  fontsize=12pt,
  a4paper,  % Standard format - only KOMAScript uses paper=a4 - https://tex.stackexchange.com/a/61044/9075
  twoside,  % we are optimizing for both screen and two-side printing. So the page numbers will jump, but the content is configured to stay in the middle (by using the geometry package)
  bibliography=totoc,
  %               idxtotoc,   %Index ins Inhaltsverzeichnis
  %               liststotoc, %List of X ins Inhaltsverzeichnis, mit liststotocnumbered werden die Abbildungsverzeichnisse nummeriert
  headsepline,
  cleardoublepage=empty,
  %               draft    % um zu sehen, wo noch nachgebessert werden muss - wichtig, da Bindungskorrektur mit drin
  draft=false
]{scrbook}
\input{include/config}

\usepackage[
  title={Fuzzy systems for Big Data computing\\\&\\Ordinary differential equations not solved for the derivative},
  author={Antonio Coín Castro},
  type={Bachelor's Thesis},
  institute={Universidad de Granada},
  course={Doble Grado en Ingeniería Informática y Matemáticas.},
  faculty={E.T.S de Ingenierías Informática y de Telecomunicación,\\Facultad de Ciencias.},
  supervisor={Margarita Arias López,\\José Manuel Benítez Sánchez.},
  enddate={September 7, 2020},
  language=english
]{\string"include/scientific-thesis-cover"}

% Hier stehen alle Abkürzungen
\newacronym{ode}{ODE}{ordinary differential equation}


%% Math environments
\theoremseparator{.}
\newtheorem{theorem}{Theorem}[chapter]
\newtheorem{corollary}[theorem]{Corollary}
\newtheorem{lemma}[theorem]{Lemma}
\newtheorem{prop}[theorem]{Proposition}

\theorembodyfont{\normalfont}
\newtheorem{definition}[theorem]{Definition}

\theoremheaderfont{\normalfont\itshape}
\newtheorem{example}[theorem]{Example}

\theoremstyle{nonumberplain}
\newtheorem{remark}[theorem]{Remark}

\renewcommand{\qed}{\blacksquare}
\theoremsymbol{\qed}
\newtheorem{proof}{Proof}

\makeindex

\begin{document}

\frontmatter
%tex4ht-Konvertierung verschönern
\iftex4ht
  % tell tex4ht to create picures also for formulas starting with '$'
  % WARNING: a tex4ht run now takes forever!
  \Configure{$}{\PicMath}{\EndPicMath}{}
  %$ % <- syntax highlighting fix for emacs
  \Css{body {text-align:justify;}}

  %conversion of .pdf to .png
  \Configure{graphics*}
  {pdf}
  {\Needs{"convert \csname Gin@base\endcsname.pdf
      \csname Gin@base\endcsname.png"}%
    \Picture[pict]{\csname Gin@base\endcsname.png}%
  }
\fi


% EN: To avoid issues with Springer's \mathplus
%     See also http://tex.stackexchange.com/q/212644/9075
\providecommand\mathplus{+}

%% Custom commands
\newcommand{\abs}[1]{\left\lvert#1\right\rvert}
\newcommand{\R}{\mathbb{R}}
\newcommand{\der}{\frac{dy}{dx}}
\newcommand{\derinv}{\frac{dx}{dy}}

%% Redefine
\renewcommand{\phi}{\varphi}
\renewcommand{\epsilon}{\varepsilon}

\pagenumbering{roman}
\Coverpage

\setlength{\parindent}{1.5em}
\linespread{1.1}

\pagestyle{empty}
\renewcommand*{\chapterpagestyle}{empty}
\Versicherung
\cleardoublepage

%Eigener Seitenstil fuer die Kurzfassung und das Inhaltsverzeichnis
\deftripstyle{preamble}{}{}{}{}{}{\pagemark}
%Doku zu deftripstyle: scrguide.pdf
\pagestyle{preamble}
\renewcommand*{\chapterpagestyle}{preamble}



%Kurzfassung / abstract
%auch im Stil vom Inhaltsverzeichnis
\ifdeutsch
  \section*{Kurzfassung}
\else
  \section*{Abstract}
\fi

<Short summary of the thesis>

\cleardoublepage

\section*{Resumen extendido en español}

<Resumen de unas 1500 palabras del trabajo>

\cleardoublepage

% BEGIN: Verzeichnisse

\iftex4ht
\else
  \microtypesetup{protrusion=false}
\fi

%%%
% Literaturverzeichnis ins TOC mit aufnehmen, aber nur wenn nichts anderes mehr hilft!
% \addcontentsline{toc}{chapter}{Literaturverzeichnis}
%
% oder zB
%\addcontentsline{toc}{section}{Abkürzungsverzeichnis}
%
%%%

%Produce table of contents
%
%In case you have trouble with headings reaching into the page numbers, enable the following three lines.
%Hint by http://golatex.de/inhaltsverzeichnis-schreibt-ueber-rand-t3106.html
%
%\makeatletter
%\renewcommand{\@pnumwidth}{2em}
%\makeatother
%
\tableofcontents

% Bei einem ungünstigen Seitenumbruch im Inhaltsverzeichnis, kann dieser mit
% \addtocontents{toc}{\protect\newpage}
% an der passenden Stelle im Fließtext erzwungen werden.

\listoffigures
%\listoftables

%Wird nur bei Verwendung von der lstlisting-Umgebung mit dem "caption"-Parameter benoetigt
%\lstlistoflistings
%ansonsten:
%\ifdeutsch
%  \listof{Listing}{Verzeichnis der Listings}
%\else
%  \listof{Listing}{List of Listings}
%\fi

%mittels \newfloat wurde die Algorithmus-Gleitumgebung definiert.
%Mit folgendem Befehl werden alle floats dieses Typs ausgegeben
%\ifdeutsch
%  \listof{Algorithmus}{Verzeichnis der Algorithmen}
%\else
%  \listof{Algorithmus}{List of Algorithms}
%\fi
%\listofalgorithms %Ist nur für Algorithmen, die mittels \begin{algorithm} umschlossen werden, nötig

% Abkürzungsverzeichnis
\printnoidxglossaries

\iftex4ht
\else
  %Optischen Randausgleich und Grauwertkorrektur wieder aktivieren
  \microtypesetup{protrusion=true}
\fi

% END: Verzeichnisse


% Headline and footline
\renewcommand*{\chapterpagestyle}{scrplain}
\pagestyle{scrheadings}
\pagestyle{scrheadings}
\ihead[]{}
\chead[]{}
\ohead[]{\headmark}
\cfoot[]{}
\ofoot[\usekomafont{pagenumber}\thepage]{\usekomafont{pagenumber}\thepage}
\ifoot[]{}


%% vv  scroll down for content  vv %%






























par
%%%%%%%%%%%%%%%%%%%%%%%%%%%%%%%%%%%%%%%%%%%%%%%%%%%%%%%%%%%%%%%%%%%%%%%%%%%%%%
%
% Main content starts here
%
%%%%%%%%%%%%%%%%%%%%%%%%%%%%%%%%%%%%%%%%%%%%%%%%%%%%%%%%%%%%%%%%%%%%%%%%%%%%%%
\mainmatter

\chapter{Introduction}
\label{chap:k1}
<Detailed introduction to the work done>


\part{Fuzzy systems for Big Data computing}

\chapter{Fuzzy set theory}
\input{\TPath/ch2}

\part{Ordinary differential equations not solved for the derivative}

\chapter{Basic concepts}
By an \textit{ordinary differential equation of order $n$} we mean a relation of the form
\begin{equation}
\label{eq:ode}
F(x, y, y', \dots, y^{(n)}),
\end{equation}
where $F$ is a real-valued function of $n+1$ real variables, $x$ is an independent variable, $y=y(x)$ is an unknown function and $y', \dots, y^{(n)}$ are the first $n$ derivatives of this function with respect to $x$. When the value of $n$ is understood or simply not relevant, we will refer to \eqref{eq:ode} only as an \gls{ode}.



%%%%%%%%%%%%%%%%%%%%%%%%%%%%%%%%%%%%%%%%%%%%%%%%%%%%%%%%%%%%%%%%%%%%%%%%%%%%%%
%
% Bibliography
%
%%%%%%%%%%%%%%%%%%%%%%%%%%%%%%%%%%%%%%%%%%%%%%%%%%%%%%%%%%%%%%%%%%%%%%%%%%%%%%

%\nocite{*}
%\printbibliography

%All links were last followed on September 7, 2020.

%\cleardoublepage

\end{document}
