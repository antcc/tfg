\documentclass{standalone}
\usepackage{pgfplots}
%\pgfplotsset{ytick style={draw=none}}
\begin{document}


\pgfmathdeclarefunction{gauss}{2}{%
  \pgfmathparse{exp(-((x-#1)^2)/(#2^2))}%
}

\pgfmathdeclarefunction{fuzzytrapezoid}{4}{%
\begingroup%
\pgfmathparse{max(min((x-#1)/(#2-#1),1,(#4-x)/(#4-#3)),0)}%
\pgfmathfloattofixed{\pgfmathresult}%
\pgfmathreturn\pgfmathresult pt\relax%
\endgroup%
}
\pgfmathdeclarefunction{fuzzytriangle}{3}{%
\begingroup%
\pgfmathparse{max(min((x-#1)/(#2-#1),(#3-x)/(#3-#2)),0)}%
\pgfmathfloattofixed{\pgfmathresult}%
\pgfmathreturn\pgfmathresult pt\relax%
\endgroup%
}
\pgfmathdeclarefunction{fuzzygaussian}{2}{%
\begingroup%
\pgfmathparse{exp(-0.5*((x-#1)/#2)^2)}%
\pgfmathfloattofixed{\pgfmathresult}%
\pgfmathreturn\pgfmathresult pt\relax%
\endgroup%
}
\pgfmathdeclarefunction{fuzzygenbell}{3}{%
\begingroup%
\pgfmathparse{1/(1+abs((x-#3)/#1)^(2*#2))}%
\pgfmathfloattofixed{\pgfmathresult}%
\pgfmathreturn\pgfmathresult pt\relax%
\endgroup%
}
\pgfmathdeclarefunction{fuzzysigmoid}{2}{%
\begingroup%
\pgfmathparse{1/(1+exp(-#1*(x-#2))}%
\pgfmathfloattofixed{\pgfmathresult}%
\pgfmathreturn\pgfmathresult pt\relax%
\endgroup%
}

\begin{tikzpicture}
\begin{axis}[xlabel={$x$},
  ylabel={$\mu(x)$},
  ytick={0,1},
  xtick=\empty,
  every axis plot post/.append style={
  mark=none,domain=-15:15,smooth},
  axis x line=bottom, 
  axis y line=left,
  x label style={at={(axis description cs:0.5,0.05)}},
  y label style={at={(axis description cs:0.05,0.5)}},
  enlargelimits=upper]
  
  \addplot[teal, smooth, samples=1000] {fuzzysigmoid(0.5,1)};
  
\end{axis}
%\draw (-0.06,0) node[left] {$0$};
\end{tikzpicture}
\end{document}